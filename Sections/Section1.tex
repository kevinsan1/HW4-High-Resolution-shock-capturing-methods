%!TEX root = /Users/kevin/SkyDrive/KTH Work/Period 3 2014/DN2255/Homework/4/HW4-High Resolution shock-capturing methods/HW4_High_Resolution_shock_capturing_methods.tex


\section{Shallow water with non-horizontal bottom} 

% (fold)
\label{sec:shallow_water_with_non_horizontal_bottom}

The shallow water model of HW2 is now extended to a non-horizontal bottom ``bathymetry'' ${B(x)}$. 
\begin{align}
	\begin{pmatrix}
		h_t\\
		hu_t 
	\end{pmatrix}
	+ 
	\begin{pmatrix}
		h_x u + h u_x\\h u u_x + gh( h_x + B_x ) 
	\end{pmatrix}
	= 
	\begin{pmatrix}
		0\\
		0 
	\end{pmatrix}
	\label{eq:originalEquation} 
\end{align}
\begin{description}
	\item[a)] Show that still water ($u=0$) must have, as it should, a horizontal water level. 
	
	If $u=0$ for all space and time, then $u_t$ and $u_x$ are also zero. Equation~\eqref{eq:originalEquation} becomes 
	\begin{equation}
		\begin{pmatrix}
			h_t\\0 
		\end{pmatrix}
		+ 
		\begin{pmatrix}
			0\\gh( h_x + B_x ) 
		\end{pmatrix}
		= 
		\begin{pmatrix}
			0\\0 
		\end{pmatrix}
		\label{eq:equationIfUiszero} 
	\end{equation}
	This means that $h_t=0$ and $h_x=-B_x$. We know that $B(x)_t=0$ because it isn't a function of time. The total height, $D(x,t)$, is $D(x,t) = h(x,t) + B(x)$. A horizontal water level means that $D(x,t) = \text{constant}$ and $D(x,t)_x=D(x,t)_t=0$. Equation~\eqref{eq:proveHorizontalWaterlevel} shows that derivatives with respect to $x$ and $t$ are zero and therefore showing that the total height $D(x,t)$ is constant. 
	\begin{align}
		D(x,t)_x &= h_x + B_x = h_x + (-h_x) = 0\\
		D(x,t)_t &= h_t + B_t = 0 + 0 = 0 \label{eq:proveHorizontalWaterlevel} 
	\end{align}
	
	\item[b)] Write the equation in conservation form for $h$ and $m=hu$. 
	\begin{equation}
		\begin{pmatrix}
			h\\m 
		\end{pmatrix}
		_t + 
		\begin{pmatrix}
			m\\f_2(h,m) 
		\end{pmatrix}
		_x = 
		\begin{pmatrix}
			0\\s(h,m,x) 
		\end{pmatrix}
		\label{eq:proveThisEquation} 
	\end{equation} % (eq:proveThisEquation)
	The first equation in Equation~\eqref{eq:proveThisEquation} is proved using the product rule, $m_x = h_x u + h u_x$. The second equation needs to be written out explicitly and solved for the unknowns. Substituting $m_t = h_t u + h u_t$ into Equation~\eqref{eq:proveThisEquation} 
	\begin{align}
		(h_t u + h u_t) + [h u u_x + gh( h_x + B_x )] = h_t u  \label{eq:firstToConservation} 
	\end{align}
	
	% (eq:firsttoconservation)
	and substituting $u=m/h$ and $h_t=-m_x$ 
	\begin{equation}
		m_t + [h \left(\frac{m}{h}\right) \left(\frac{m}{h}\right)_x + gh( h_x + B_x )] = (-m_x) \left(\frac{m}{h}\right) \label{eq:secondtoConservation} 
	\end{equation}
	
	% (eq:secondtoconservation)
	and bringing x and t derivatives together and putting derivatives other than $h$ or $m$ to the source function 
	\begin{align}
		m_t + [\underbrace{m \left(\frac{m}{h}\right)_x + m_x \left(\frac{m}{h}\right)}_{(m^2/h)_x} + \underbrace{gh h_x}_{(gh^2/2)_x}] &= -gh B_x \label{eq:thirdConservation} \\
			m_t + (\underbrace{\frac{m^2}{h} + \frac{gh^2}{2}}_{f_2(h,m)})_x &= \underbrace{-gh B_x}_{s(h,m,x)}\label{eq:fourthConservation}
	\end{align} % (eq:thirdconservation)
	Equation~\eqref{eq:proveThisEquation} written in conservation form becomes,
\begin{equation}
	\begin{pmatrix}
		h\\m 
	\end{pmatrix}
	_t + 
	\begin{pmatrix}
		m\\\frac{m^2}{h} + \frac{gh^2}{2} 
	\end{pmatrix}
	_x = 
	\begin{pmatrix}
		0\\ -g h B_x
	\end{pmatrix}
	\label{eq:proveThisEquationSolved}
\end{equation} % (eq:provethisequationsolved)
\end{description}

% section shallow_water_with_non_horizontal_bottom (end)
